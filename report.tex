\documentclass[a4paper,11pt]{article}

\usepackage[T1]{fontenc}
\usepackage[utf8]{inputenc}
\usepackage{graphicx}
\usepackage{xcolor}

% --- Font Settings ---
\renewcommand\familydefault{\sfdefault}
\usepackage{tgheros}
% zi4 is a high-quality, professional monospaced font available in almost all distributions
\usepackage[varqu]{zi4} 

\usepackage{amsmath,amssymb,amsthm,textcomp}
\usepackage{geometry}
\geometry{left=25mm,right=25mm, top=20mm,bottom=20mm}

\linespread{1.3}
\newcommand{\linia}{\rule{\linewidth}{0.5pt}}

% --- Header and Footer ---
\usepackage{fancyhdr}
\pagestyle{fancy}
\lhead{}
\chead{}
\rhead{}
\lfoot{Assignment Report}
\cfoot{}
\rfoot{Page \thepage}
\renewcommand{\headrulewidth}{0pt}

% --- Code Listing Settings ---
\usepackage{listings}
\lstset{
    language=Go,
    basicstyle=\ttfamily\small,
    aboveskip={1.0\baselineskip},
    belowskip={1.0\baselineskip},
    columns=fixed,
    extendedchars=true,
    breaklines=true,
    tabsize=4,
    frame=lines,
    keywordstyle=\color[rgb]{0.627,0.126,0.941},
    commentstyle=\color[rgb]{0.133,0.545,0.133},
    stringstyle=\color[rgb]{0,0,0},
    numbers=left,
    numberstyle=\small,
    stepnumber=1,
    numbersep=10pt,
    captionpos=t
}

\begin{document}

\begin{center}
\vspace{2ex}
{\huge \textsc{Tournament Algorithm: 3-Process Solution}}
\vspace{1ex}
\\
\linia\\
Alessandro Monticelli, Jóhannes Helgi Tómasson, Jonas Stahl \hfill \today % Replace with your name
\vspace{4ex}
\end{center}

\section*{Description of Solution}
The solution is derived from tracing the logic of the Bankers algorithm, as found at \\ https://github.com/mkyas/banker/blob/main/banker.go \\
The initial variables are taken from the assignment description for the Bankers algorithm on Canvas and listed as the following:
\begin{lstlisting}[caption=Initial variables for the problem]
total <Plates, Bowls> : 8, 12
request <Plates, Bowls> {
  [P1] : 7, 7
  [P2] : 6, 10
  [P3] : 1, 2
  [P4] : 2, 4
}
allocation <Plates, Bowls> {
  [P1] : 2, 3
  [P2] : 3, 5
  [P3] : 0, 1
  [P4] : 1, 2
}
need <Plates, Bowls> {
  [P1] : 5, 4
  [P2] : 3, 5
  [P3] : 1, 1
  [P4] : 1, 2
}
available <Plates, Bowls> : 2, 1
\end{lstlisting}

\newpage
\section*{A. Will the restaurant be able to feed all four parties successfully? }
% Describe your solution here (100-300 words).
In the first pass of the bankers algorithm, a loop runs for each table, checking if the availability of resources is greater or equal to their need. If not the ok flag is set to false and they wait. In the first pass, only P3 continues, the request is granted such that:
\begin{lstlisting}[caption=Output of variables after running grant\_request]
    available      = [2,1] - [1,2] = [1,-1]
    allocation[P3] = [0,1] + [1,2] = [1, 3]
    need[P3]       = [1,1] - [1,2] = [0,-1]
\end{lstlisting}
We go into the \textbf{Safe} function and make a copy of the available array, into free. And check, in a loop, if any tables need can be satisfied. We see P3 can be satisfied and add its allocation to free.
\textbf{free} $= [1,-1] + [1, 3] = [2,2]$.
P3 is marked as done, and the loop resets, free can now satisify P4, free increases to [3,4] and the loop resets. No other tables can be satisfied, so \textbf{Safe} returns false, and the previous state is restored.
The restaurant couldn't feed all parties.


\section*{B. Assume a new dinner party,  P5, comes to the restaurant now. Their maximum needs are five plates and three bowls. Initially, the waiter brings two plates to them. To feed all five parties successfully, what is the minimum number of additional bowls and plates the restaurant needs to buy?}
\begin{itemize}
    \item The restaurant must first buy a single plate to serve P3. Free becomes [1,2] 
    \item They can serve P4, so free becomes [2,4]. The restaurant can't serve more parties.
    \item They buy an extra plate. So free becomes [3,4]
    \item They serve P5, free becomes [5,4].
    \item They serve P1, free becomes [7,7].
    \item They serve P2, free becomes [10,12]
    \item The restaurant only need to purchase two additional plates.

\end{itemize}

\end{document}
